% Options for packages loaded elsewhere
\PassOptionsToPackage{unicode}{hyperref}
\PassOptionsToPackage{hyphens}{url}
%
\documentclass[
]{article}
\usepackage{lmodern}
\usepackage{amssymb,amsmath}
\usepackage{ifxetex,ifluatex}
\ifnum 0\ifxetex 1\fi\ifluatex 1\fi=0 % if pdftex
  \usepackage[T1]{fontenc}
  \usepackage[utf8]{inputenc}
  \usepackage{textcomp} % provide euro and other symbols
\else % if luatex or xetex
  \usepackage{unicode-math}
  \defaultfontfeatures{Scale=MatchLowercase}
  \defaultfontfeatures[\rmfamily]{Ligatures=TeX,Scale=1}
\fi
% Use upquote if available, for straight quotes in verbatim environments
\IfFileExists{upquote.sty}{\usepackage{upquote}}{}
\IfFileExists{microtype.sty}{% use microtype if available
  \usepackage[]{microtype}
  \UseMicrotypeSet[protrusion]{basicmath} % disable protrusion for tt fonts
}{}
\makeatletter
\@ifundefined{KOMAClassName}{% if non-KOMA class
  \IfFileExists{parskip.sty}{%
    \usepackage{parskip}
  }{% else
    \setlength{\parindent}{0pt}
    \setlength{\parskip}{6pt plus 2pt minus 1pt}}
}{% if KOMA class
  \KOMAoptions{parskip=half}}
\makeatother
\usepackage{xcolor}
\IfFileExists{xurl.sty}{\usepackage{xurl}}{} % add URL line breaks if available
\IfFileExists{bookmark.sty}{\usepackage{bookmark}}{\usepackage{hyperref}}
\hypersetup{
  pdftitle={Statistical Inference Course Project},
  pdfauthor={Gyulben},
  hidelinks,
  pdfcreator={LaTeX via pandoc}}
\urlstyle{same} % disable monospaced font for URLs
\usepackage[margin=1in]{geometry}
\usepackage{color}
\usepackage{fancyvrb}
\newcommand{\VerbBar}{|}
\newcommand{\VERB}{\Verb[commandchars=\\\{\}]}
\DefineVerbatimEnvironment{Highlighting}{Verbatim}{commandchars=\\\{\}}
% Add ',fontsize=\small' for more characters per line
\usepackage{framed}
\definecolor{shadecolor}{RGB}{248,248,248}
\newenvironment{Shaded}{\begin{snugshade}}{\end{snugshade}}
\newcommand{\AlertTok}[1]{\textcolor[rgb]{0.94,0.16,0.16}{#1}}
\newcommand{\AnnotationTok}[1]{\textcolor[rgb]{0.56,0.35,0.01}{\textbf{\textit{#1}}}}
\newcommand{\AttributeTok}[1]{\textcolor[rgb]{0.77,0.63,0.00}{#1}}
\newcommand{\BaseNTok}[1]{\textcolor[rgb]{0.00,0.00,0.81}{#1}}
\newcommand{\BuiltInTok}[1]{#1}
\newcommand{\CharTok}[1]{\textcolor[rgb]{0.31,0.60,0.02}{#1}}
\newcommand{\CommentTok}[1]{\textcolor[rgb]{0.56,0.35,0.01}{\textit{#1}}}
\newcommand{\CommentVarTok}[1]{\textcolor[rgb]{0.56,0.35,0.01}{\textbf{\textit{#1}}}}
\newcommand{\ConstantTok}[1]{\textcolor[rgb]{0.00,0.00,0.00}{#1}}
\newcommand{\ControlFlowTok}[1]{\textcolor[rgb]{0.13,0.29,0.53}{\textbf{#1}}}
\newcommand{\DataTypeTok}[1]{\textcolor[rgb]{0.13,0.29,0.53}{#1}}
\newcommand{\DecValTok}[1]{\textcolor[rgb]{0.00,0.00,0.81}{#1}}
\newcommand{\DocumentationTok}[1]{\textcolor[rgb]{0.56,0.35,0.01}{\textbf{\textit{#1}}}}
\newcommand{\ErrorTok}[1]{\textcolor[rgb]{0.64,0.00,0.00}{\textbf{#1}}}
\newcommand{\ExtensionTok}[1]{#1}
\newcommand{\FloatTok}[1]{\textcolor[rgb]{0.00,0.00,0.81}{#1}}
\newcommand{\FunctionTok}[1]{\textcolor[rgb]{0.00,0.00,0.00}{#1}}
\newcommand{\ImportTok}[1]{#1}
\newcommand{\InformationTok}[1]{\textcolor[rgb]{0.56,0.35,0.01}{\textbf{\textit{#1}}}}
\newcommand{\KeywordTok}[1]{\textcolor[rgb]{0.13,0.29,0.53}{\textbf{#1}}}
\newcommand{\NormalTok}[1]{#1}
\newcommand{\OperatorTok}[1]{\textcolor[rgb]{0.81,0.36,0.00}{\textbf{#1}}}
\newcommand{\OtherTok}[1]{\textcolor[rgb]{0.56,0.35,0.01}{#1}}
\newcommand{\PreprocessorTok}[1]{\textcolor[rgb]{0.56,0.35,0.01}{\textit{#1}}}
\newcommand{\RegionMarkerTok}[1]{#1}
\newcommand{\SpecialCharTok}[1]{\textcolor[rgb]{0.00,0.00,0.00}{#1}}
\newcommand{\SpecialStringTok}[1]{\textcolor[rgb]{0.31,0.60,0.02}{#1}}
\newcommand{\StringTok}[1]{\textcolor[rgb]{0.31,0.60,0.02}{#1}}
\newcommand{\VariableTok}[1]{\textcolor[rgb]{0.00,0.00,0.00}{#1}}
\newcommand{\VerbatimStringTok}[1]{\textcolor[rgb]{0.31,0.60,0.02}{#1}}
\newcommand{\WarningTok}[1]{\textcolor[rgb]{0.56,0.35,0.01}{\textbf{\textit{#1}}}}
\usepackage{graphicx,grffile}
\makeatletter
\def\maxwidth{\ifdim\Gin@nat@width>\linewidth\linewidth\else\Gin@nat@width\fi}
\def\maxheight{\ifdim\Gin@nat@height>\textheight\textheight\else\Gin@nat@height\fi}
\makeatother
% Scale images if necessary, so that they will not overflow the page
% margins by default, and it is still possible to overwrite the defaults
% using explicit options in \includegraphics[width, height, ...]{}
\setkeys{Gin}{width=\maxwidth,height=\maxheight,keepaspectratio}
% Set default figure placement to htbp
\makeatletter
\def\fps@figure{htbp}
\makeatother
\setlength{\emergencystretch}{3em} % prevent overfull lines
\providecommand{\tightlist}{%
  \setlength{\itemsep}{0pt}\setlength{\parskip}{0pt}}
\setcounter{secnumdepth}{-\maxdimen} % remove section numbering

\title{Statistical Inference Course Project}
\author{Gyulben}
\date{17 11 2020 г}

\begin{document}
\maketitle

Part 1

1.1. Overview In this project we will investigate the exponential
distribution in R and compare it with the Central Limit Theorem. The
exponential distribution can be simulated in R with rexp(n, lambda)
where lambda is the rate parameter. The mean of exponential distribution
is 1/lambda and the standard deviation is also 1/lambda. lambda = 0.2
for all of the simulations.

1.2. Simulations

\begin{Shaded}
\begin{Highlighting}[]
\CommentTok{#pre-defined parameters}
\NormalTok{lambda <-}\StringTok{ }\FloatTok{0.2}
\NormalTok{n <-}\StringTok{ }\DecValTok{40}
\NormalTok{sims <-}\StringTok{ }\DecValTok{1}\OperatorTok{:}\DecValTok{1000}
\KeywordTok{set.seed}\NormalTok{(}\DecValTok{123}\NormalTok{)}

\CommentTok{# Check for missing dependencies and load necessary R packages}
\ControlFlowTok{if}\NormalTok{(}\OperatorTok{!}\KeywordTok{require}\NormalTok{(ggplot2))\{}\KeywordTok{install.packages}\NormalTok{(}\StringTok{'ggplot2'}\NormalTok{)\}; }\KeywordTok{library}\NormalTok{(ggplot2)}
\end{Highlighting}
\end{Shaded}

\begin{verbatim}
## Loading required package: ggplot2
\end{verbatim}

\begin{verbatim}
## Warning: package 'ggplot2' was built under R version 3.6.3
\end{verbatim}

\begin{Shaded}
\begin{Highlighting}[]
\CommentTok{# Simulate the population}
\NormalTok{population <-}\StringTok{ }\KeywordTok{data.frame}\NormalTok{(}\DataTypeTok{x=}\KeywordTok{sapply}\NormalTok{(sims, }\ControlFlowTok{function}\NormalTok{(x) \{}\KeywordTok{mean}\NormalTok{(}\KeywordTok{rexp}\NormalTok{(n, lambda))\}))}

\CommentTok{# Plot the histogram}
\NormalTok{hist.pop <-}\StringTok{ }\KeywordTok{ggplot}\NormalTok{(population, }\KeywordTok{aes}\NormalTok{(}\DataTypeTok{x=}\NormalTok{x)) }\OperatorTok{+}\StringTok{ }
\StringTok{  }\KeywordTok{geom_histogram}\NormalTok{(}\KeywordTok{aes}\NormalTok{(}\DataTypeTok{y=}\NormalTok{..count.., }\DataTypeTok{fill=}\NormalTok{..count..)) }\OperatorTok{+}
\StringTok{  }\KeywordTok{labs}\NormalTok{(}\DataTypeTok{title=}\StringTok{"Histogram for Averages of 40 Exponentials over 1000 Simulations"}\NormalTok{, }\DataTypeTok{y=}\StringTok{"Frequency"}\NormalTok{, }\DataTypeTok{x=}\StringTok{"Mean"}\NormalTok{)}
\NormalTok{hist.pop}
\end{Highlighting}
\end{Shaded}

\begin{verbatim}
## `stat_bin()` using `bins = 30`. Pick better value with `binwidth`.
\end{verbatim}

\includegraphics{Part-1_files/figure-latex/unnamed-chunk-1-1.pdf}

1.3. Sample Mean versus Theoretical Mean

\begin{Shaded}
\begin{Highlighting}[]
\CommentTok{# the Sample Mean & Theoretical Mean}
\NormalTok{sample.mean <-}\StringTok{ }\KeywordTok{mean}\NormalTok{(population}\OperatorTok{$}\NormalTok{x)}
\NormalTok{theoretical.mean <-}\StringTok{ }\DecValTok{1}\OperatorTok{/}\NormalTok{lambda}
\KeywordTok{cbind}\NormalTok{(sample.mean, theoretical.mean)}
\end{Highlighting}
\end{Shaded}

\begin{verbatim}
##      sample.mean theoretical.mean
## [1,]    5.011911                5
\end{verbatim}

\begin{Shaded}
\begin{Highlighting}[]
              \DecValTok{5}
\end{Highlighting}
\end{Shaded}

\begin{verbatim}
## [1] 5
\end{verbatim}

\begin{Shaded}
\begin{Highlighting}[]
\CommentTok{# Checking 95% confidence interval for Sample Mean}
\KeywordTok{t.test}\NormalTok{(population}\OperatorTok{$}\NormalTok{x)}
\end{Highlighting}
\end{Shaded}

\begin{verbatim}
## 
##  One Sample t-test
## 
## data:  population$x
## t = 204.53, df = 999, p-value < 2.2e-16
## alternative hypothesis: true mean is not equal to 0
## 95 percent confidence interval:
##  4.963824 5.059998
## sample estimates:
## mean of x 
##  5.011911
\end{verbatim}

1.4. Sample Variance versus Theoretical Variance

\begin{Shaded}
\begin{Highlighting}[]
\NormalTok{sample.variance <-}\StringTok{ }\KeywordTok{var}\NormalTok{(population}\OperatorTok{$}\NormalTok{x)}
\NormalTok{theoretical.variance <-}\StringTok{ }\NormalTok{((}\DecValTok{1}\OperatorTok{/}\NormalTok{lambda)}\OperatorTok{^}\DecValTok{2}\NormalTok{)}\OperatorTok{/}\NormalTok{n}
\KeywordTok{cbind}\NormalTok{(sample.variance, theoretical.variance)}
\end{Highlighting}
\end{Shaded}

\begin{verbatim}
##      sample.variance theoretical.variance
## [1,]       0.6004928                0.625
\end{verbatim}

1.5. Distribution

\begin{Shaded}
\begin{Highlighting}[]
\CommentTok{# Plotting Sample Mean & Varience vs Theoretical Mean & Varience}
\NormalTok{gg <-}\StringTok{ }\KeywordTok{ggplot}\NormalTok{(population, }\KeywordTok{aes}\NormalTok{(}\DataTypeTok{x=}\NormalTok{x)) }\OperatorTok{+}
\StringTok{  }\KeywordTok{geom_histogram}\NormalTok{(}\KeywordTok{aes}\NormalTok{(}\DataTypeTok{y=}\NormalTok{..density.., }\DataTypeTok{fill=}\NormalTok{..density..)) }\OperatorTok{+}
\StringTok{  }\KeywordTok{labs}\NormalTok{(}\DataTypeTok{title=}\StringTok{"Histogram of Averages of 40 Exponentials over 1000 Simulations"}\NormalTok{, }\DataTypeTok{y=}\StringTok{"Density"}\NormalTok{, }\DataTypeTok{x=}\StringTok{"Mean"}\NormalTok{) }\OperatorTok{+}\StringTok{ }
\StringTok{  }\KeywordTok{geom_density}\NormalTok{(}\DataTypeTok{colour=}\StringTok{"blue"}\NormalTok{) }\OperatorTok{+}
\StringTok{  }\KeywordTok{geom_vline}\NormalTok{(}\DataTypeTok{xintercept=}\NormalTok{sample.mean, }\DataTypeTok{colour=}\StringTok{"blue"}\NormalTok{, }\DataTypeTok{linetype=}\StringTok{"dashed"}\NormalTok{) }\OperatorTok{+}
\StringTok{  }\KeywordTok{stat_function}\NormalTok{(}\DataTypeTok{fun=}\NormalTok{dnorm,}\DataTypeTok{args=}\KeywordTok{list}\NormalTok{( }\DataTypeTok{mean=}\DecValTok{1}\OperatorTok{/}\NormalTok{lambda, }\DataTypeTok{sd=}\KeywordTok{sqrt}\NormalTok{(theoretical.variance)),}\DataTypeTok{color =} \StringTok{"red"}\NormalTok{) }\OperatorTok{+}
\StringTok{  }\KeywordTok{geom_vline}\NormalTok{(}\DataTypeTok{xintercept=}\NormalTok{theoretical.mean, }\DataTypeTok{colour=}\StringTok{"red"}\NormalTok{, }\DataTypeTok{linetype=}\StringTok{"dashed"}\NormalTok{) }
\NormalTok{gg}
\end{Highlighting}
\end{Shaded}

\begin{verbatim}
## `stat_bin()` using `bins = 30`. Pick better value with `binwidth`.
\end{verbatim}

\includegraphics{Part-1_files/figure-latex/unnamed-chunk-4-1.pdf}

\end{document}
